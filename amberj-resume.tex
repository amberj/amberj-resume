\documentclass[margin,line]{resume}
\usepackage{hyperref}

\begin{document}
\name{\Large Amber Jain's \Resume \footnote{The latest version of this resume is available at \url{http://amberj.devio.us/resume.html}}}
\begin{resume}

    % Contact Information    
    \section{\mysidestyle Contact\\Information}\vspace{2mm}
    \begin{tabular}{@{} l @{\hspace{69mm}} r}
    24-B, Siddharth Nagar & (+91)8982732654 \\
    Behind Gurjar Hospital, Bhanwarkua & i.amber.jain@gmail.com \\
    Indore (452001) MP, INDIA & http://amberj.devio.us/ \\
    \end{tabular}

    % Academic/Research Interests
    \section{\mysidestyle Interests}

    Computational Intelligence, Machine Learning, Unix-like systems, Social Network Analysis.

    % Education \\ *- ONGOING
    \section{\mysidestyle Education}
    
    \begin{list2}
	\item \textbf{Master of Computer Applications} \hspace{73mm} \textbf{CGPA: 7.48/10} \\ International Institute of Professional Studies, Devi Ahilya University \hspace{17mm} July 2010 - June 2013
	\end{list2}

    \begin{list2}
	\item \textbf{Introduction to Databases} \hspace{80mm} \textbf{ 	225.0 out of 293.0} \\ Class2Go \hspace{110mm} January - March 2013
    \end{list2}

    \begin{list2}
	\item \textbf{Networked Life} \hspace{120mm} \textbf{95.6\%} \\ Coursera \hspace{105mm} September - October 2012
	\end{list2}
    
    \begin{list2}
	\item \textbf{Computing for Data Analysis} \hspace{70mm} \textbf{99\% with Distinction} \\ Coursera \hspace{105mm} September - October 2012
	\end{list2}

    \begin{list2}
	\item \textbf{Writing in the Sciences} \hspace{78mm} \textbf{97.8\% with Distinction} \\ Coursera \hspace{102mm} September - November 2012
	\end{list2}
    
    \begin{list2}
	\item \textbf{Learn to Program: The Fundamentals} \hspace{80mm} \textbf{77.8\%} \\ Coursera \hspace{102mm} September - November 2012
	\end{list2}
    
    \begin{list2}
	\item \textbf{Introduction to Computer Science} (CS101) \hspace{50mm} \textbf{Highest Distinction} \\ Udacity \hspace{112mm} February - April 2012
	\end{list2}

    \begin{list2}
	\item \textbf{Machine Learning} \hspace{59mm} \textbf{77/80 (HW), 800/800 (Programming)} \\ Coursera \hspace{105mm} October - December 2011
	\end{list2}
    
    \begin{list2}
	\item \textbf{Introduction to Artificial Intelligence} (CS271) \hspace{70mm} \textbf{91.7\%} \\ Udacity \hspace{108mm} October - December 2011
	\end{list2}
    
	\begin{list2}
	\item \textbf{Bachelor of Computer Applications} \hspace{70mm} \textbf{CGPA: 7.55/10} \\ International Institute of Professional Studies, Devi Ahilya University \hspace{20mm} July 2007 - June 2010
	\end{list2}

    % Professional Experience
    \section{\mysidestyle Employment/ \\ Activities}

    \begin{list2}
	\item \textbf{Google Summer of Code 2012 Intern (OpenCog Foundation)} \hspace{15mm} May 2012 - August 2012 \\ OpenCog packages for Ubuntu (and other Unix-like systems) and Cygwin port.
	\item \textbf{Volunteer sysadmin and developer (Development Center, IIPS)} \hspace{12mm} September 2012 - now \\ Volunteer system administrator and developer at e-governance center at IIPS, Devi Ahilya University.
	\end{list2}

    % Skills
    \section{\mysidestyle Skills} 

    \begin{list2}
	\item \textbf{Programming/Tools}: \hspace{11.8mm} Python, MATLAB/Octave, JavaScript, HTML, CSS, C, bash
	\item \textbf{Databases/Serialization}: \hspace{5.7mm} SQL, MySQL, SQLite,	JSON
	\item \textbf{Operating Systems}: \hspace{13.8mm} Unix-like (Ubuntu and Debian), Windows
	\item \textbf{Version Control Systems}: \hspace{3.5mm} Git, Mercurial, Bazaar, Subversion
	\item \textbf{Build tools}: \hspace{28mm} Cygwin, make, cmake
	\item \textbf{Cloud and Virtualization}: \hspace{3mm} Eucalyptus, VirtualBox
	\end{list2}

    % Projects
    \section{\mysidestyle Projects} 
	\begin{list2}
	\item \textbf{ocpkg}: Scripts for packaging, deploying and managing OpenCog instances.
	\item \textbf{ubudeb-dl}: A Python program to download .deb package (and it's dependencies) from packages.ubuntu.com even without the need of an Ubuntu installation.
	\item \textbf{udacity-cs101}: Implementation of a barebones search engine in Python. I wrote this for Udacity's CS101: Intro to CS: Building a Search Engine course.
	\item \textbf{ml-class}: Octave code for Prof. Andrew Ng's Stanford online Machine Learning class.
	\item \textbf{Investigation of Symmetric Block Cipher Algorithms}: Academic project in which I investigated state-of-art symmetric block cipher algorithms.
    \end{list2}

	% Publications
	\section{\mysidestyle Publications}
	\begin{list2}
	\item \textbf{Windows Autorun FAQs}: A four-part article series about autoruns and autostart location in Windows (published on http://bytes.com/)
	\end{list2}

    % Miscellaneous
    \section{\mysidestyle Personal \\ Information}
    \begin{list2}
    \item Languages: Hindi and English
    \item Preferences: I'm willing to work remotely or relocate to a new city/country.
    \item Citizen: India
    \item USA Immigration status: I need a visa to work in USA.
    \end{list2}

    % References
    \section{\mysidestyle References} 

    \begin{list2}
    \item {\sl Shaligram Prajapat} (Associate Professor, International Institute of Professional Studies, Devi Ahilya University), shaligram.prajapat@gmail.com
	\item {\sl Vivek Shrivastava} (Assistant Professor, International Institute of Professional Studies, Devi Ahilya University), shrivastava.vivex@gmail.com
    \end{list2}

\end{resume}
\end{document}
